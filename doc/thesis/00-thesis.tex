\documentclass[11pt, rgb, bibtotoc, twoside]{scrreprt}

\usepackage{themeKonstanz} 

\usepackage{lipsum}

\format{a4}

% Thesis information  %
\year{2024}
\author{Fabian Klopfer, M.Sc.}
\title{Extraction of Quantitative Relaxation and Diffusion Metrics from balanced Steady-State Free Precession Magnetic Resonance Imaging}
%Multi-dimensional Feature Extraction of Quantitative Relaxation and Diffusion Brain Magnetic Resonance Imaging
\uni{University of Tübingen}
\unisection{Faculty of Science \\ Faculty of Medicine}
\department{Graduate School of Neuroscience}
\institute{Department for High-field Magnetic Resonance \\ Max Planck Institute for Biological Cybernetics}
\supervisorOne{Dr. Rahel Heule}
\supervisorTwo{Prof. Dr. Klaus Scheffler}

\headFoot{14}
 
\bibliography{resources} 

\begin{document}
\thesistitlepage[language=english]

\begin{abstract}
\begin{center}
    \textbf{Abstract:}
\end{center}{}
\lipsum[1]
\end{abstract}


\newgeometry{left=2.5cm, right=2.5cm, bottom=2cm, top=2.5cm, headheight=14pt, headsep=0.8cm, footskip=30pt}
\tableofcontents


\restoregeometry
\rmfamily 
\normalsize

\chapter{Introduction}
\begin{itemize}
 \item Largest multi modal data set of its kind
 \item bSSFP now feasible with new HW
 \item Speeds up scanning times potentially yielding multiple modalities with one scan
 \item Limited data available for ML applications
\end{itemize}

\cleardoublepage
\chapter{Related Work}
\section{Brain Imaging DataStructure}
what why what it looks like

\section{MRI Processing/Registration Tools}
FSL, SPM, ANTS, FreeSurfer

\section{Quantitative MRI using bSSFP}
Some bla bla about qMRI development in the past. \\
In depth description of sequence, profile, banding
\subsection{Configuration Modes \& Asymmetry Maps}
karla miller paper

\subsection{T1 and T2 Map Estimation using MIRACLE}
TESS \& MIRACLE papers

\section{Deep Learning in Medical Imaging and MRI}
UNet, challenges, impact of transformers, ...


\chapter{Methods}\label{\positionnumber} 
\section{The Dove Dataset}
Some stats on size 
\subsection{Participants \& Study Design}
briefly describe participants, task, scanning times (of the day)

\subsection{Recorded Sequences}
cf gais doc.
Details on acquisitions

\section{Processing Pipeline}
requirements
\subsection{bSSFP}
\subsection{Phase Correction}
Why, how

\subsubsection{Motion Correction \& Co-Registration}
\paragraph{Initial FSL Pipeline} motion correction to first. Direct, inverse to t1w, B1, T1w rescaling. Difficulties with artefacts, special requirements due to quanti, auto scaling of registered image

\paragraph{SPM Pipeline} exact steps that spm executes. difficulties with docs, lack of API


\subsection{DWI}
TODO ask svenja what was done


\section{Machine Learning-based DWI Tensor estimation from bSSFP data}
MONAI, torchio, pytorch lightning. 

\subsection{Baseline}
UNet + some extra layers to fit output dim. 

\subsection{Augmentation}

\subsection{PreTraining}
Sparse data, small batch sizes. => Augmentation not enough => Pre training
\paragraph{AutoEncoder PreTraining}
As auto encoder. train autoenc for bssfp only 

\paragraph{ExtraHead Transfer and Fine tuning}
freeze autoenc, add head. transfer then unfreeze all and fine tune
Unet + extra head

\subsection{Complex vs. non-complex}
input and weights 










\cleardoublepage
\chapter{Results}\label{\positionnumber}

\section{Direct bSSFP to DWI Tensor Training}

\section{Pre-Training \& bSSFP Transfer-Learning}

\section{Comparison of Voxel-wise to Volume-wise Regression}

\section{Modality Comparisons}
\subsection{Quantitative comparison of Quality differences between Auto-Encoding and pc-bSSFP Image to Image Translation}
\subsection{pc-bSSFP is more suitable than single-volume Structural Images}
\subsection{Phase-cycles contain relevant the Information}
\subsection{Asymmetry Index Maps distill phase-cycle Information}








\cleardoublepage
\chapter{Discussion}\label{\positionnumber}

\section{Limitations}
voxel- patch - volume pro-con\\
Statistical power \\
model size/speed \\
learned vs. closed-form/deduced/analytical transformation \\
igh frequency features and gan loss (show diff maps) \\
Augmentation \\
error propagation of denorm to scalars \\
masks \& probsegs imperfect, currently regenerated w freesurfer by svenja k \\
registration for bssfp for many subject difficult \\

\section{Future Work}
\subsection{Modular Deep Learning}
Pretrain .\\
LDM for latent translation\\
GAN \& Stable diffusion, control flow matching, latent diffusion models\\
phase cycle und miller \\
predict t2ws to establish better comparability wrt. scalars \\
multi res, multi input, multi output/foundation model \\

\subsection{Conv Transf.}
New kid on the block\\

\subsection{Neural Arch Search}
DiNTS

\chapter{Acknowledgements}
Rahel, Flo, Qi, Klaus

\cleardoublepage
\appendix
\renewcommand{\thesection}{\Alph{section}}
\chapter*{Appendix}
\addcontentsline{toc}{chapter}{Appendix}

\section{Supplementary Material}

   \begin{figure}[h]
      \begin{center}
         %[keepaspectratio, width=0.45\textwidth]{img/0.jpg}
      \end{center}
      \caption{
         \textbf{A. \& B.} SNc and projections to the dorsal striatum in healthy subjects and patients with PD.
         In the healthy subject, the SNc is still highly pigmented due to the melanin-containing dopamine-producing cells being intact.
         The SNc projects to the striatum and delivers normal amounts of DA into the basal ganglia (BG) circuit.
         If the dopamine-producing neurons undergo apoptosis, the pigmentation decreases and so does the amount of dopamine adimnistered to the striatum.
         \textbf{C.} Photomicrographs of Lewy bodies.
      }
      \label{broad-mech}
   \end{figure}

\printbibliography




   \begin{figure}[h]
      \begin{center}
         %[keepaspectratio, width=0.45\textwidth]{img/0.jpg}
      \end{center}
      \caption{
         \textbf{A. \& B.} SNc and projections to the dorsal striatum in healthy subjects and patients with PD. 
         In the healthy subject, the SNc is still highly pigmented due to the melanin-containing dopamine-producing cells being intact.
         The SNc projects to the striatum and delivers normal amounts of DA into the basal ganglia (BG) circuit.
         If the dopamine-producing neurons undergo apoptosis, the pigmentation decreases and so does the amount of dopamine adimnistered to the striatum. 
         \textbf{C.} Photomicrographs of Lewy bodies.
      }
      \label{broad-mech}
   \end{figure}


\section{Discussion}




\end{document}
